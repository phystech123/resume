\documentclass[a4paper, 12pt]{article}
\usepackage[english, russian]{babel}
\usepackage[T2A]{fontenc}
\usepackage[utf8]{inputenc}

\usepackage{amsmath}
\usepackage{amsfonts}
\usepackage{amssymb}
\usepackage{mathtools}
\usepackage{amsthm}

\theoremstyle{plain}
\newtheorem{theorem}{Теорема}
\newtheorem{lemma}{Лемма}

\usepackage{indentfirst}
\usepackage{soulutf8}
\usepackage{amsfonts, amssymb}

\usepackage{geometry}
\geometry{top=25mm}
\geometry{bottom=30mm}
\geometry{left=20mm}
\geometry{right=20mm}

\usepackage{titleps}
\newpagestyle{main}{
    \setheadrule{0.4pt}
    \sethead{}{}{}
    \setfootrule{0.4pt}
    \setfoot{}{\thepage}{}
}
\usepackage{enumitem}
\usepackage{xcolor}
\usepackage{hyperref}
\usepackage{wrapfig}
\usepackage{graphics}

\usepackage{tikz-cd}
\usepackage{tikz}
\usepackage{tikz-3dplot}
\usetikzlibrary{calc}


\usepackage{caption}
\usepackage{subcaption}
\usepackage[demo]{graphicx}

% % from gnuplot site:
% \usepackage{amsmath,amssymb,amsfonts,textcomp,latexsym,pb-diagram,amsopn}
% \usepackage{cite,enumerate,float,indentfirst}
% \usepackage{graphicx,xcolor}

% \usepackage{gnuplottex}
\usepackage{pgfplots}
\usepackage{pgfplotstable}
\pgfplotsset{width=10cm,compat=newest}

\usepackage{pdfpages}

\renewcommand{\phi}{\varphi}
\renewcommand{\epsilon}{\varepsilon}
\renewcommand{\kappa}{\varkappa}
% \usepackage{mathastext} - обычный шрифт в формулах



\usepackage{booktabs}

\usepackage{gensymb}

\begin{document}
\section{Личная информация}
\begin{itemize}
    \item ФИО: Рябов Олег Евгеньевич
    \item Телефон: 89801883670
    \item Email: riabov.oe@phystech.edu
    \item Telegram: @tinkof7
    \item Город: Долгопрудный
    \item Портфолио: в репозитории
    \item Цель: участие в "Школе математического моделирования в фармацевтике"
\end{itemize}	

\section{Образование}
\begin{itemize}
    \item ВУЗ: Московский физико-технический институт (МФТИ)
    \item Факультет: Физтех-школа электроники, фотоники и молекулярной физики (ФЭФМ)
    \item Направление: Прикладная маематика и физика (ПМФ)
    \item Степень обучения: бакалавриат
    \item Год окончания: 2027 
\end{itemize}

\section{Дополнительное образование}
\begin{itemize}
    \item Курс математического моделирования от МФТИ (2024-2025г)
    \item "A/B week" Yandex (2025г)
    \item Курс "Машинное обучение в атомистическом моделировании" от Сколтеха (2025г)
    \item Школа по ML в биоинформатике от ФКН ВШЭ (2025г)
    \item Курс по хемоинформатике Незлобин МФТИ (2025г)
    \item Курс ОИЯИ по математическому моделированию систем (2025г)
    \item Курс от НГУ по Ansys (2025г)
    \item КУрс от ИТМО по хемоинформатике (2025г)
    \item Курс введение в научную деятельонсть от МФТИ (2025г)
    \item Курс Физика карьеры от Ассоциации Карьерных Консультантов (2025г)
\end{itemize}

\section{Опыт работы}
\begin{itemize}
    \item Центр вычислительной физики МФТИ (менторский проект), рассчет углов смачивания (since 2025)
    \item Преподавание в Очной Физико-Технической школе при МФТИ (информатика, математика) (since 2025)
    \item СКАТ (Студенческий конкурс авиационного творчества от ФАЛТ МФТИ, вожатый) (2024, 2025)
    \item Приемная комиссия ФЭФМ МФТИ (представлял факультет) (2024)
\end{itemize}

\section{Навыки}
\begin{itemize}
    \item Программа МФТИ по математическим дисциплинам:
    \begin{itemize}
        \item Математический анализ/теория функций комплексного переменного
        \item Аналитическая геометрия/линейная алгебра
        \item Теория вероятности/математическая статистика
        \item Машинное обучение
        \item Дифференциальные уравнения/уравнения математической физики
        \item Вычислительная математика
    \end{itemize}
    \item Программа МФТИ по общей физике и теорфизике, химии и химической физике
    \item IT инструменты
    \begin{itemize}
        \item C++ (годовой курс МФТИ)        - уверенно
        \item Python (годовой курс МФТИ)     - уверенно
        \item Git                            - уверенно
        \item Linux (Ubuntu/Manjaro)         - уверенно
        \item Html, css, js, php             - базовый уровень 
        \item SQL (в том числе через Python) - базовый уровень
        \item Docker                         - базовый уровень
    \end{itemize}
    \item Библиотеки Python
    \begin{itemize}
        \item numpy
        \item pandas
        \item scipy
        \item simpy
        \item matplotlib
        \item seaborn 
        \item bokeh
        \item torch (pytorch)
        \item aiogram
    \end{itemize}
    \item Инструменты для моделирования
    \begin{itemize}
        \item lammps
        \item gromacs
        \item phonopy
        \item rdkit
        \item material project
        \item mattersim
        \item martini
        \item ovito
    \end{itemize}
    \item Языки:
    \begin{itemize}
        \item Русский    - носитель
        \item Английский - B1/B2
    \end{itemize}
\end{itemize}

\section{Soft skills}
\begin{itemize}
    \item усидчивость
    \item концентрация
    \item исполнительность
    \item креативность
\end{itemize}

\end{document}